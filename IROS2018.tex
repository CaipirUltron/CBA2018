\documentclass[letterpaper, 10 pt, conference]{ieeeconf}  % Comment this line out if you need a4paper

%\documentclass[a4paper, 10pt, conference]{ieeeconf}      % Use this line for a4 paper

\IEEEoverridecommandlockouts                              % This command is only needed if
                                                          % you want to use the \thanks command

\overrideIEEEmargins                                      % Needed to meet printer requirements.

% See the \addtolength command later in the file to balance the column lengths
% on the last page of the document

\usepackage[latin1]{inputenc}
%\usepackage[brazil]{babel}
%\usepackage[T1]{fontenc}
\usepackage{ae}

\makeatletter
\def\verbatim@font{\normalfont\ttfamily\footnotesize}
\makeatother

\usepackage{graphics} % for pdf, bitmapped graphics files
\usepackage{epsfig} % for postscript graphics files
%\usepackage{mathptmx} % assumes new font selection scheme installed
%\usepackage{times} % assumes new font selection scheme installed
\usepackage{amsmath}
\usepackage{amssymb}  % assumes amsmath package installed
%\let\proof\relax
%\let\endproof\relax
%\usepackage{amsthm}
\usepackage{color}
%\usepackage{mathtools}
\usepackage{tabularx} %used to chose the widht of tables
\usepackage{multirow}
\usepackage{epstopdf}
\let\labelindent\relax
\usepackage[inline]{enumitem}

\newtheorem{remark}{Coment�rio}
\theoremstyle{plain}
\newtheorem{definition}{Defini��o}
\newtheorem{teorema}{Teorema}
\newtheorem{algorithm}{Algoritmo}
\newcommand{\sref}[1]{Se��o~\ref{#1}}
\newcommand{\fref}[1]{Fig.~\ref{#1}}
\newcommand{\tref}[1]{Tabela~\ref{#1}}
\newcommand{\norm}[1]{\left\lVert#1\right\rVert}
\newcommand{\aref}[1]{Suposi��o~\ref{#1}}
\newcommand{\algref}[1]{\textbf{Algoritmo~\ref{#1}}}
%\renewcommand{\qedsymbol}{}
\newcommand{\rev}[1]{{\color{red}#1}}
\newcommand{\mat}[1]{\begin{bmatrix}#1\end{bmatrix}}
 
%%%%%%%%%%%%%%%%%%%%%%%%%%%%%%%%%%%%%%%%%%%%%%%%%%%%%%%%%%%%%%%%%%%%%%%%%%%%%%%%%%%%%%%%%%%%%%%%%%%%%%%%%%%%%%%%%%%%%%%%%%%%%%%%%%%%%
\title{\LARGE \bf Identification and Feedback Linearization Control of a 3-DOF Inertial Platform for Line of Sight Stabilization and Tracking}

\author{Matheus F. Reis, Jo\~ao C. Monteiro, Guilherme P. S. Carvalho, Alex F. Neves and Alessandro J. Peixoto% <-this % stops a space
%\thanks{*This work was supported by Repsol Sinopec Brasil under the ANP-Brazil R\&D program (project reference number 20005-5)}% <-this % stops a space
%
%\thanks{The authors are with the Electrical Engineering Department. of COPPE/UFRJ, Rio de Janeiro, Brazil
%{\tt\small (matheus.ferreira.reis @gmail.com).}}}
%
\thanks{*The authors are with the Electrical Engineering Department of COPPE/UFRJ, Rio de Janeiro, Brazil {\tt\small \{matheus.ferreira.reis, guilherme.ps.carvalho, alexfneves\}@gmail.com, jcmonteiro@poli.ufrj.br, jacoud@coep.ufrj.br.}}}

%%%%%%%%%%%%%%%%%%%%%%%%%%%%%%%%%%%%%%%%%%%%%%%%%%%%%%%%%%%%%%%%%%%%%%%%%%%%%%%%%%%%%%%%%%%%%%%%%%%%%%%%%%%%%%%%%%%%%%%%%%%%%%%%%%%%%
\begin{document}

\maketitle
\thispagestyle{plain}
\pagestyle{plain}

%% Abstract section
\twocolumn[
\maketitle
%\selectlanguage{english}

\begin{abstract}
The majority of works in line of sight (LOS) stabilization and tracking using inertially stabilized platforms (ISP) apply simple linear controllers to achieve the required performance. Commonly, linear models are employed to describe the relationship between torque and position of the ISP joints, such as a double integrator with an inertia gain.
%
However, high-accuracy and fast motion applications may require more complex control techniques, demanding accurate dynamic and kinematic models, and system \textit{identification} procedures.
%
In this work, we propose a cascade control topology for stabilization and tracking of the line of sight of a camera in a 3-DOF ISP installed on a vessel. Using measurements from joint encoders and an inertial navigation system (INS) fixed on the vessel, an inner controller cancels the \textit{dynamic} disturbances acting on the ISP joints, while an outer kinematic controller ensures LOS tracking.
%
For this type of controller, the ISP joint axes are critical parameters with respect to the tracking performance, since their uncertainty introduce bias in the control response. For this reason, a simple, yet efficient identification procedure for the joint axes is proposed.
%
Simulations in Gazebo using the Rock robotics framework and a Unity-based viewer show the efficiency of this procedure and the performance of the proposed cascade controller for LOS stabilization and tracking.
\end{abstract}

\keywords{Line of sight stabilization, inertial platform, computed torque control.}
%%%%%%%%%%%%%%%%%%%%%%%%%%%%
\selectlanguage{brazil}

\begin{abstract}
A maioria dos trabalhos em estabiliza��o e rastreamento da linha de visada (LOS) com plataformas inerciais (ISPs) utilizam simples controladores lineares para atingir o n�vel de performance exigido. Geralmente, modelos lineares s�o suficientes para descrever a rela��o �ngulo-torque para as juntas da plataforma, tais como um simples duplo integrador.
%
Por�m, em opera��es que requerem alta velocidade e/ou acur�cia, t�cnicas de controle mais complexas podem ser necess�rias. Tais t�cnicas em geral demandam modelos mais realistas para o sistema considerado, bem como procedimentos para a identifica��o de par�metros destes modelos.
%
Neste trabalho, � proposta uma topologia de controle em cascata para estabiliza��o e rastreamento da linha de visada de uma c�mera, utilizando uma plataforma inercial de 3 eixos instalada em um navio.
%
Utilizando medi��es de encoders localizados nas juntas e um sistema de navega��o inercial (INS) acoplado ao navio, um controlador interno cancela perturba��es din�micas devido ao movimento do navio, enquanto um controlador cinem�tico externo garante o rastreamento da LOS.
%
Neste abordagem, os eixos das juntas s�o par�metros cr�ticos em rela��o ao desempenho de rastreamento, pois sua incerteza introduz um bias em rela��o a refer�ncia desejada.
%
Por isso, um simples e eficiente procedimento de identifica��o � proposto.
%
Simula��es em Gazebo com o ambiente de simula��o Rock e um visualizador baseado em Unity mostram a efici�ncia do procedimento e o desempenho do controlador proposto.
\end{abstract}

\keywords{Estabiliza��o da linha de visada, plataformas inerciais, controle por torque computado.}
]
\selectlanguage{brazil}
\input{introduction}
\input{modeling_NE}
\input{problem_formulation}
\input{control}

\documentclass[
    tikz,
    border={0pt 0pt 20pt 0pt}, % left bottom right top
]{standalone}
\usepackage[T1]{fontenc}
\usepackage[utf8]{inputenc}
\usepackage{pgfplots}
\usepackage{grffile}
\pgfplotsset{compat=newest}
\usetikzlibrary{plotmarks}
\usetikzlibrary{arrows.meta}
\usepgfplotslibrary{patchplots}
\usepackage{amsmath}

\begin{document}
%
\begin{tikzpicture}

\begin{axis}[%
width=9cm,
height=4cm,
at={(0.772in,0.484in)},
scale only axis,
xmin=0,
xmax=210,
xlabel={Time (seconds)},
xtick={0,40,...,200},
ytick={-60,-30,...,60},
ymin=-80,
ymax=80,
title={Angle (deg)},
title style={at={(0.08,0.96)}, anchor=south west, align=left},
xmajorgrids,
ymajorgrids,
legend columns=3,
legend style={at={(1.0,1.03)}, anchor=south east, legend cell align=left, align=left, draw=white!15!black}
]
\addplot [solid, color=blue]
  table[]{./data/identification-1.tsv};
\addlegendentry{joint 1}

\addplot [dash dot, color=red]
  table[]{./data/identification-2.tsv};
\addlegendentry{joint 2}

\addplot [dashed, color=black]
  table[]{./data/identification-3.tsv};
\addlegendentry{joint 3}

\end{axis}
\end{tikzpicture}%
\end{document}

\input{simulations}
\input{conclusion}
\input{acknowledgements}

%A conclusion section is not required. Although a conclusion may review the main points of the paper, do not replicate the abstract as the conclusion. A conclusion might elaborate on the importance of the work or suggest applications and extensions.

\addtolength{\textheight}{-12cm}  % This command serves to balance the column lengths
                                  % on the last page of the document manually. It shortens
                                  % the textheight of the last page by a suitable amount.
                                  % This command does not take effect until the next page
                                  % so it should come on the page before the last. Make
                                  % sure that you do not shorten the textheight too much.

%%%%%%%%%%%%%%%%%%%%%%%%%%%%%%%%%%%%%%%%%%%%%%%%%%%%%%%%%%%%%%%%%%%%%%%%%%%%%%%%%%%%%%%%%%%%%%%%%%%%%%%%%%%%%%%%%%%%%%%%%%%%%%%%%%%%%
%\section*{APPENDIX}
%
%\section*{ACKNOWLEDGMENT}
%%%%%%%%%%%%%%%%%%%%%%%%%%%%%%%%%%%%%%%%%%%%%%%%%%%%%%%%%%%%%%%%%%%%%%%%%%%%%%%%%%%%%%%%%%%%%%%%%%%%%%%%%%%%%%%%%%%%%%%%%%%%%%%%%%%%%
\bibliographystyle{IEEEtran}
\bibliography{IROS2018}

\end{document} 