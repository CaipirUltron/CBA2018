\begin{abstract}

%The majority of works in line of sight (LOS) stabilization and tracking using inertially stabilized platforms (ISP) apply simple linear controllers to achieve the required performance. Commonly, linear models are employed to describe the relationship between torque and position of the ISP joints, such as a double integrator with an inertia gain.
%%
%However, high-accuracy and fast motion applications may require more complex control techniques, demanding accurate dynamic and kinematic models, and system \textit{identification} procedures.
%%
%In this work, we propose a cascade control topology for stabilization and tracking of the line of sight of a camera in a 3-DOF ISP installed on a vessel. Using measurements from joint encoders and an inertial navigation system (INS) fixed on the vessel, an inner controller cancels the \textit{dynamic} disturbances acting on the ISP joints, while an outer kinematic controller ensures LOS tracking.
%%
%For this type of controller, the ISP joint axes are critical parameters with respect to the tracking performance, since they introduce bias in the control response. For this reason, a simple, yet efficient identification procedure for the joint axes is proposed.
%%
%Simulations in Gazebo using the Rock robotics framework and a Unity-based viewer show the efficiency of this procedure and the performance of the proposed cascade controller for LOS stabilization and tracking.

A maioria dos trabalhos em estabiliza��o da linha de visada ()

\end{abstract}